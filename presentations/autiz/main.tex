\documentclass{beamer}

\mode<presentation>{
%\usecolortheme{beaver}
%\usecolortheme{crane}
}
\usepackage[utf8]{inputenc}
\usepackage{default}
\usepackage{pgfplots}
\pgfplotsset{/pgf/number format/use comma,compat=newest}
\usepackage{xcolor}
\usepackage{amsmath,amsfonts,amssymb}
\usepackage{hyperref}
\usepackage{tikz}

\useoutertheme[subsection=false]{miniframes}

\definecolor{pene}{rgb}{.125,.4,.25}
\usecolortheme[named=pene]{structure}

\title{Raising capital for an gold mine (operational in 2021) through Tokenization}
\subtitle{Securing investor trust through Tezos}
\author[Dinkar Ganti, Shekar Mantha \& Jovan Smith] {Dinkar Ganti, Shekar Mantha \& Jovan Smith}
\institute[Autiz LLC]

\begin{document}
\begin{frame}
 \maketitle
\end{frame}

%\begin{frame}
% \tableofcontents
%\end{frame}
%\I have a sale pending for 50% of the mine asset base for $10mm. The mine is going back into production in 2021. We have approximately 3000 acres of placer and hard rock gold claims. Insitu gold deposit of approximately one billion. Approximately $1mm is setting on the property in improvements and equipment. Hard to estimate production. We have approximately 1 gram per ton of substrate. A lot of substrate has to be processed hourly to be economical. Contiguous to my mining property, they are processing approximately 5000 ton daily and recovering an average of 22 oz AU. at 75% pure daily. This is now about $28k.
%\

\section{Objective}
\begin{frame}
\frametitle{Objective}
\begin{minipage}{\textwidth}
	\begin{itemize}
    \item  Raise 1MM USD to bring a mine back to production in 2021.
  \end{itemize}

\end{minipage}
\end{frame}

\begin{frame}
\frametitle{Roles and responsibilities}
\begin{minipage}{\textwidth}
  \begin{itemize}
    \item  Entity with placer and hard rock gold claims.
    \item  Autiz LLC - Entity with responsibility to manage the tokenization.
  \end{itemize}
\end{minipage}
\end{frame}

\begin{frame}
\frametitle{Details about the Mine}
\begin{minipage}{\textwidth}
  \begin{itemize}
    \item  Mine is pending sale for 50 \% of the mine asset base for USD 10MM.
    \item  Mine is not currently operational (estimated to be in 2021).
    \item  Mine has an \textit{in situ} gold deposit of approximately USD 1,000.00 MM
    \item  Mine needs USD 1MM for improvements in property and equipment (more detail needed).
    \item  Estimated daily production (based on neighboring contiguous property) - 22 oz at 75 \% purity.
  \end{itemize}
\end{minipage}
\end{frame}

\begin{frame}
\frametitle{Tokenization process and details}
\begin{minipage}{\textwidth}
  \begin{itemize}
    \item  Divide the tokenization into phases of USD 100,000.00 per round (in steps of three.) Each tokenization phase last between 30 - 90 days.
    \item  Establish contracts to monitor the progress of the use of funds.
    \item  Create a dividend schedule based on the daily production.
  \end{itemize}
  \end{minipage}
\end{frame}

\begin{frame}
\frametitle {Tokenization details per round}
\begin{minipage}{\textwidth}
  \begin{itemize}
    \item  1 USD = 100 AUTIZ tokens.
    \item  Total percentage of the mine to be tokenized is about 10 \% ~ USD 100.00 MM in the lifetime of the mine.
    \item  AUTIZ commission = 0.001 \% AUTIZ value.
    \item  Production for the mine is captured in a Production contract tied to the Mine.
    \item  An audit smart contract that validates the financials of the asset holding company (aka the Mine).
    \item  AUTIZ tokens represent the valuation of the production of the mine and \textit{not} the value of gold (AU) at any given day.
  \end{itemize}
  \end{minipage}
\end{frame}

\begin{frame}
  \frametitle {Contracts and Details}
  \begin{minipage}{\textwidth}
    \begin{itemize}
    \item  Tokenization contract - the core contract establishing the value of the token.
    \item  Production schedule contract - A contract backed by a mobile application to record and publish mine production figures.
    \item  Auditor's contract - A contract that validates financials every quarter.
    \item  A dividend report contract - Performs the actual transfer of dividend based on the quarterly results.
    \item  A contract to compute penalties and fines that the underlying asset needs to pay for non-performance.
    \end{itemize}
  \end{minipage}
\end{frame}

\begin{frame}
  \frametitle {Tokenization Roadmap}
  \begin{minipage}{\textwidth}
    \begin{itemize}
    \item  Private, by invitation only, tokenization. This has the advantage of KYC/AML requirements as well as accreditation requirements for the token.
    For example, each investor can only buy in 20,000 AUTIZ lots. Another advantage of this approach is building credibility on the street. Though, tokenization is not sufficient to build such trust and the gaps in credibility needs to be filled through adherence to the letter of the local laws and make them binding through the relevant smart contracts.
    \item  Secutirized token
      We would propose to securitize the token after the third of fourth round of tokenization to ensure trust.
    \end{itemize}
  \end{minipage}
\end{frame}

\begin{frame}
  \frametitle {Product roadmap}
  \begin{minipage}{\textwidth}
    \begin{itemize}
      \item  Build the financial model
      \item  Setup the legal structure
      \item  Setup a team to build and release the product.
    \end{itemize}
  \end{minipage}
\end{frame}

\begin{frame}
  \frametitle {Questions and comments}
  \begin{minipage}{\textwidth}
    \begin{itemize}
      \item  Questions and comments.
    \end{itemize}
  \end{minipage}
\end{frame}

\end{document}
