\documentclass{beamer}
\usepackage{graphicx}
\usepackage{hyperref}
\newcommand{\quotes}[1]{``#1''}
\graphicspath{ {./images/} }
\title{Properties contaminated with oil - investment scenarios}
\author{Dinkar Ganti}
\date {\today}
\begin{document}
\begin{frame}
  \titlepage
\end{frame}
\begin{frame}
\frametitle{Tasks}
\section{Tasks}
  The number of tasks to cleanup a property can spread out over weeks or months and therefore it is important to list them and manage them. Here we list a subset of tasks that need to be caputured in the current real world.
  \begin{itemize}
    \item Baseline analysis or site characterization --- estimated between 5000 - 20000 USD.
    \item Case manager's assessment of the property to provide cleanup guidance.
    \item Environment consultant's cleanup plan.
    \item Environment consultant's rate list as per DEP's desirable rate list per task. 
    \item Installation of cleanup and monitoring equipment.
    \item Maintainence of currency of insurance documents.
    \item Workers liability
    \item Attorneys signoff for task if needed.
    \item Escrow accounts : in real life such account are needed. (note: in a trustless environment these accounts are simpler to setup).
  \end{itemize}
\end{frame}
\begin{frame}
\frametitle {Glossary}
\section{Glossary}
As in any field the field of contamination remediation is full of jargon that can be mind-numbing at times. Here we try to cover some of the terms to help bridge the gap.
\begin{itemize}
  \item DEP --- The department of environment protection is a state agency in the United States that has complete jurisdiction on the cleanup plan process and approval of an tasks that need to be completed as part of the cleanup. 
  \item NFA --- Short for No Further Action. This document usually ties to any lien the DEP may have on the property. An NFA implies that the property is free from contamination and the relevant case is closed. For the investor it means that the property can be resold at prices comparable to similar properties in the neighborhood.
  \item USEPA --- United States Environment Protection Agency. An agency with a global directive on environment law covering soil, water and air within the 48 states, Alaska and Hawaii. Despite such a title, the USEPA's involvement in properties is limited for most properties partly due to states rights (need to confirm this) and when they do get involved the sites are usually declared as brownfields and have a different source of funding as well as jurisdiction, which we will cover in a later section \hyperref{brownfields} on Brownfields.
\end{itemize}
\end{frame}

\begin{frame}
\frametitle{Where is the opportunity ? (not relevant for the dApp)}

\section {Opportunity}

Usually these properties are written-off by the property owners as the costs of cleanup for even a relatively minor spill can go upwards of USD 40000 (empirical evidence). The purpose of realitz is to bring together innovative solution providers with investors to realize the potential increase in the value of a property contaminated with oil after an effective cleanup. Formula for evaluating the potential increase of the value of a property needs to account for a few principles, namely the value of the property after cleanup is capped by the comparables in the market and the risk of cleanup needs to be assumed to be infinite, analogous to the risk of an option writer in the regulated options exchange. 

In other words, the users of this application will be a community of solution providers with sufficient experience and a trackrecord as well as investors and other parties involved. It is likely that investors will lose their investment (full disclosure). The purpose of this application is not to discuss the merits of an investment or a solution. The purpose of this application is to present with an immutable ledger of events and a place to contractually manage tasks involving their properties. The application provides for a way to communicate the status of a property.

\end{frame}

\begin{frame}
\frametitle{Brownfields}
\section{Brownfields} \label{brownfields}
   \href{https://www.epa.gov/brownfields/overview-brownfields-program} {Brownfields program} is supported by the \href{https://www.epa.gov/} {USEPA} to help property owners or counties that need to redevelop properties that are contaminated and where the cleanup can be quite complicated as well as expensive. The USEPA provides grants to property owners to help cleanup the property and encourage developers to take the risk needed to redevelop the property. Brownfields present a great opportunity as there is shared risk between the USEPA and the property owner. 
\end{frame}

\begin{frame}
\frametitle{Solution Providers}

\section{Solution Providers}

Realitz is partly funded by \href{https://www.sarvabioremed.com}{Sarva Bio Remed} through an exclusive partnership between Realitz (structure to be decided) and Sarva Bio Remed LLC to help complete all of the cleanup using the solutions provided by Sarva Bio Remed LLC. We believe that this partnership sets Realitz apart from other solution providers because the cost of cleanup is significantly reduced. Recently, Dinkar Ganti, one of the co-founders of Realitz and Sarva Bio Remed LLC collaborated on cleaning a property contaminated with TCE/PCE and have presented a paper \href{https://aehsconference.pathable.co/}{36th Annual International Conference on Soils, Sediments, Water and Energy} that presents the experience in greater detail. TLDR: the cleanup cost under USD 10,000.00 and the carcinogenic risk on the employees at the strip mall was significantly reduced. There were some other lessons learned that we believe could have been avoided had we put it a structure such as realitz. Namely, the time for the option was short and did not take into account delays due to regulatory reasons or \textit{force majeure} due to the COVID-19 pandemic. 

\end{frame}

\begin{frame}
\frametitle{Lessons Learned }

\section{Lessons Learned at 2331 E. Market Street, York}

The seller did experience a seller's remorse as the cost of cleanup was estimated at least 400\% more than what it took for the group performing the cleanup. The costs were near USD 50,000 and the option was priced at 225,000 less than the market price (pre COVID-19). The PCE/TCE numbers were well below range and the property would have been closer to attaining a closure from the PaDEP based on the numbers at the property. We feel that the business model did play out well, though the founders had to let the option expire because they could not raise the funds to acquire the property.
The technical details of the cleanup are presented in the blog here: \href{https://dservgun.github.io/}{Enhanced Aerobic Bio-remediation of Chlorinated Solvents using VaporRemed - Field Report}. 
\end{frame}

\begin{frame}
\frametitle{Twitter thread on cleanup of chlorinated solvents}

\section{Cleanup of chlorinated solvents and the opportunities they present for the not so faint of heart}

A twitter thread on the \href{https://twitter.com/dganti1/status/1254433456462979078}{MEW superfund site} - Mountain View Superfund Site. This superfund site was in the news because some of the employees from Googleplex were impacted with elevated levels of TCE/PCE in indoor air. The thread presents our perspective on the results, specially as we juxtapose them with the data collected at the strip mall in York, Pa.

\end{frame}

\begin{frame}
\frametitle{TLDR}
\section{TLDR}
  When a release is reported for a property that triggers a set of actions by parties involved due to the liabilities that are ensuant to the release. For example, if the spill is greater than 50 gallons in the state of Virginia, the release is accompanied with a case number to track the status and progress of the release and ensuing cleanup activities.

  For example,the image shown above describes the closure of a dep case file and the reasons for the closure. More details are outlined in the actual report though, for the purposes of this discussion we the limit our attention to the phrase \quotes{according to state law}, as this encompasses the level of scrutiny each property receives from the relevant DEP (Department of Environment Protection, usually a state regulatory body). 

\end{frame}

\begin{frame}
\frametitle{Know your client}
\section {KYC}
KYC for this application needs to add an extra layer of security if needed so that each aspect of the transaction is approved and the transfer of funds is traceable.
\end{frame}

\begin{frame}

\frametitle{Rules}
\section{Rules}
  There are various rules and validations that need to be in place for a task to be completed. For example, 
    \begin{itemize}
        \item Each cleanup task for a property needs to be associated with a Case Number. 
        \item Each case should be actively handled by a case manager.
        \item All contractors need to have valid licenses.
        \item All workers on site need to have their insurances in order.
        \item All tasks that are being executed on a property need to have authorizations from
          \begin{itemize}
            \item CFO for the Property.
            \item Case Manager from the DEP.
            \item Insurance company. 
          \end{itemize}
    \end{itemize}
\end{frame}

\begin{frame}
\frametitle{TLDR2}
\section {TLDR2}
  \includegraphics[width = 4in, height = 4in]{vadeqnfa}
\end{frame}
\end{document}